%%
% The BIThesis Template for Graduate Thesis
%
% Copyright 2020-2023 Yang Yating, BITNP
%
% This work may be distributed and/or modified under the
% conditions of the LaTeX Project Public License, either version 1.3
% of this license or (at your option) any later version.
% The latest version of this license is in
%   https://www.latex-project.org/lppl.txt
% and version 1.3 or later is part of all distributions of LaTeX
% version 2005/12/01 or later.
%
% This work has the LPPL maintenance status `maintained'.
%
% The Current Maintainer of this work is Feng Kaiyu.

\begin{conclusion}

\begin{table}[htbp]
	\centering
	\caption{Comparison Among This Research And Current Works}
	\label{tab:5.14}
	\begin{tabular*}{\textwidth}{@{\extracolsep{\fill}}ccccc}
		\toprule
		Work & Virus Type & Attack Type & Dection Model Amount for Analysis & Average Evasion Rate \\
		\midrule
		{[22]} & PE Malware & black-box attack & 1 & 60.0\%(MalConv network) \\
		{[68]} & Android Malware & white-box attack & 10 & 59.37\%(Average) \\
		{[69]} & PE Malware & black-box attack & 3 & 70.0\%(AV for commercial) \\
		{[26]} & PE Malware & black-box attack & 5 & 74.4\%(EMBER model) \\
		{[70]} & ELF Malware & black-box attack & 64 & 75.8\%(VirusTotal) \\
		This research & ELF Malware & white-box attack & 62 & 84.5\%(VirusTotal) \\
		\bottomrule
	\end{tabular*}
\end{table}

Experimental results indicate that adversarial samples processed by perturbation exhibit extremely lower overall detection rates on VirusTotal than original samples, exhibiting strong transferability. Structural perturbations achieved the most stable evasion, with multiple static analysis-dependent engines failing to identify malicious features. Instructional perturbations showed limited evasion across engines, yielding relatively overall smaller variations. Behavioral perturbations effectively evaded engines with weaker dynamic detection capabilities, but there still exist partial detections in engines supporting sandbox analysis. The limitation shows platform-specific dependencies in dynamic feature interference.

Furthermore, different perturbation methods affected engine types distinctly. Structural perturbation primarily disrupted engines relying on static structure parsing and feature extraction. Instructional perturbation impacted detectors based on instructional signatures or machine learning models. Behavioral perturbation more readily evaded dynamic mechanisms based on sandbox execution paths and behavioral patterns. These results further validate the adaptability and practical utility of adversarial perturbations in multi-engine, cross-platform detection environments.

The adversarial samples generated in this research not only achieve effective evasion in local detection models but also demonstrate strong transferability in real detection platforms, exhibiting robust cross-platform escape effects. This provides solid support for adversarial malware research and practical applications.

To systematically analyze the differences and advantages between the proposed method in this experiment and existing research, Table \ref{tab:5.14} compares current typical adversarial malware sample generation works with this research across four key dimensions: virus types, attack methods, number of detection models, and average evasion rate. This table qualitatively and quantitatively describes the comprehensive performance of this work.

In the virus type aspect, most existing studies concentrate primarily on PE viruses in Windows platforms or Android viruses in mobile platforms, with relatively scarce research on in the ELF format malware in Linux platforms. In contrast, this study focuses on adversarial sample generation and escape effect evaluation for ELF virus realm. This research effectively fills this research gap of ELF adversarial malware generation.

Comparing attack methods, the approach issued in this paper belongs to the white-box attack category, like the method in literature \cite{rathore2021identification}. The white-box attack requires knowledge of internal information such as the feature extraction methods and training data distribution of the detection model. Different from this research, some literatures \cite{kolosnjaji2018adversarial,quertier2022merlin,song2022mab} employs a black-box attack pattern. Unlike white-box attack, black-box attack eliminates dependency on the internal structure of detection models and optimizes disturbances solely through input and output results.

In Comparison, another method for automatically generating ELF adversarial malware samples based on RL\cite{xue2024reinforcement} incorporates a multi-loop mechanism combining multi-round feature extraction, malware detection, and intelligent decision-making. Utilizing the PPO algorithm, it achieves effective perturbation and evasion of ELF malware samples on the Linux x86 platform.

Although both this method and the method issued in research adopts reinforcement learning frameworks, the present work focuses more on multidimensional large-scale evaluation, covering 62 detection engines to emphasize broad applicability and robustness. Simultaneously, improvements in action space design, feature selection, and attack strategies have been updated. This innovation results in a higher average evasion rate and reflects the competitive advantage of the approach in practical applications.

In terms of average evasion rate, the adversarial samples generated in this research achieved 84.5\% evasion rate on the VirusTotal platform, surpassing the average levels reported in other studies. This indicates that the proposed adversarial perturbation strategy demonstrates stronger effectiveness in evading detection.
\end{conclusion}
