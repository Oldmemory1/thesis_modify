\chapter{讨论}

未来的研究可进一步探索更加复杂和细粒度的对抗样本生成策略,结合先进的智能算法与高效的数据处理技术,以应对日益演化的网络安全威胁。同时,针对具体应用场景,提升生成效率、降低资源开销,将是推动该技术实用化的关键方向。此外,该框架也有潜力应用于其他类型的恶意软件检测与防御研究中,进一步拓展其实用范围与研究深度。

%Future research may explore more complex and fine-grained adversarial sample generation strategies, utilizing advanced intelligent algorithms and efficient data processing techniques to address evolving cybersecurity threats. Concurrently, improving generation efficiency and reducing resource expenditure for specific application scenarios will be the crucial directions for advancing the practical deployment of this technology. This framework holds potential for other malware detection and defense research domains, expanding its applicability and research depth.